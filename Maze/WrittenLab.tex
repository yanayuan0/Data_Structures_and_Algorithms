% This LaTeX file contains your written lab questions.  You may answer
% these questions just by inserting your answer into this document.
%
% If you're unfamiliar with LaTeX, see the document LearningLaTeX.tex
% in this same directory.  It contains a brief explanation and a few
% snippets of LaTeX code to get you started; in fact, it should have
% everything you need to complete this assignment.
\documentclass{article}

\usepackage{amsmath}
\usepackage{amssymb}
\usepackage{amsthm}
\usepackage{algpseudocode}
\usepackage{algorithmicx}
\usepackage{alltt}
\usepackage{enumerate}

\newtheorem{claim}{Claim}

\begin{document}
    \section{Inductive Proofs}

    Prove each of the following claims by induction

    \begin{claim}
      The sum of the first $n$ even numbers is $n^2 + n$.  That is, $\sum\limits_{i=1}^{n} 2i = n^2 + n$.
    \end{claim}

      %%%%%%%%%%%%%%
      %%%% TODO %%%%
      %%%%%%%%%%%%%%
      % Write your proof of the above statement here.
      
      \vspace{1mm}
      1. Base case: $i = 1$

      LHS: $2 * (1) = 2$

      RHS: $(1)^2 + 1 = 2$

      Thus, LHS = RHS.

      \vspace{5mm}
      2. Inductive Hypothesis: 

      Assume that $\sum\limits_{i=1}^{n} 2i = n^2 + n$ for all $1 \leqslant n \leqslant k$.

      \vspace{5mm}
      3. Inductive Step: 

      \begin{math}
        \begin{array}{r c l}
          \textnormal{Show that } \sum\limits_{i=1}^{k+1} 2i & = & (k+1)^2+(k+1) \\
                                                             & = & k^2+2k+1+(k+1) \\
                                                             & = & k^2+3k+2 \\
        \end{array}
      \end{math}

      \begin{math}
        \begin{array}{r c l}
          \sum\limits_{i=1}^{k+1} 2i & = & \sum\limits_{i=1}^{k} 2i + 2(k+1)\\
                                     & = & k^2 + k + 2(k+1)\\
                                     & = & k^2 + k + 2k + 2\\
                                     & = & k^2 + 3k + 2\\
        \end{array}
      \end{math}

      Thus, we have shown that $\sum\limits_{i=1}^{k+1} 2i = (k+1)^2+(k+1)$.
      

    \begin{claim}
      $\sum\limits_{i=1}^{n} \frac{1}{2^i} = 1 - \frac{1}{2^n}$
    \end{claim}

      %%%%%%%%%%%%%%
      %%%% TODO %%%%
      %%%%%%%%%%%%%%
      % Write your proof of the above statement here.

      \vspace{1mm}
      1. Base case: $i = 1$

      LHS: $\frac{1}{2^(1)} = \frac{1}{2}$

      RHS: $1 - \frac{1}{2^(1)} = 1 - \frac{1}{2} = \frac{1}{2}$

      Thus, LHS = RHS.

      \vspace{5mm}
      2. Inductive Hypothesis: 

      Assume that $\sum\limits_{i=1}^{n} \frac{1}{2^i} = 1 - \frac{1}{2^n}$ for all $1 \leqslant n \leqslant k$.

      \vspace{5mm}
      3. Inductive Step: 

      \begin{math}
        \begin{array}{r c l}
          \textnormal{Show that } \sum\limits_{i=1}^{k+1} \frac{1}{2^i} & = & 1 - \frac{1}{2^{(k+1)}} \\
                                                                        & = & 1 - \frac{1}{2^k} * \frac{1}{2} \\
        \end{array}
      \end{math}

      \begin{math}
        \begin{array}{r c l}
          \sum\limits_{i=1}^{k+1} \frac{1}{2^i} & = & \sum\limits_{i=1}^{k} \frac{1}{2^i} + \frac{1}{2^{(k+1)}} \\
                                                & = & 1 - \frac{1}{2^k} + \frac{1}{2^k} * \frac{1}{2} \\
                                                & = & 1 - \frac{1}{2} * \frac{1}{2^k} \\
                                                & = & 1 - \frac{1}{2^{(k+1)}} \\
        \end{array}
      \end{math}

      Thus, we have shown that $\sum\limits_{i=1}^{k+1} \frac{1}{2^i} = 1 - \frac{1}{2^{(k+1)}}$.


    \begin{claim}
      $\sum\limits_{i=0}^{n} 2^i = 2^{n+1} - 1$
    \end{claim}

      %%%%%%%%%%%%%%
      %%%% TODO %%%%
      %%%%%%%%%%%%%%
      % Write your proof of the above statement here.

      \vspace{1mm}
      1. Base case: $i = 0$

      LHS: $2^0 = 1$

      RHS: $2^{(0+1)} - 1 = 2 - 1 = 1$

      Thus, LHS = RHS.

      \vspace{5mm}
      2. Inductive Hypothesis: 

      Assume that $\sum\limits_{i=0}^{n} 2^i = 2^{n+1} - 1$ for all $0 \leqslant n \leqslant k$.

      \vspace{5mm}
      3. Inductive Step: 

      \begin{math}
        \begin{array}{r c l}
          \textnormal{Show that } \sum\limits_{i=0}^{k+1} 2^i & = & 2^{(k+1)+1} - 1 \\
                                                              & = & 2^{(k+2)} - 1 \\
        \end{array}
      \end{math}

      \begin{math}
        \begin{array}{r c l}
          \sum\limits_{i=0}^{k+1} 2^i & = & \sum\limits_{i=0}^{k} 2^i + 2^{(k+1)}\\
                                      & = & 2^{(k+1)} - 1 + 2^{(k+1)}\\
                                      & = & 2^k * (2+2) - 1\\
                                      & = & 2^k * 4 - 1\\
                                      & = & 2^k * 2^2 - 1\\
                                      & = & 2^{(k+2)} - 1\\
        \end{array}
      \end{math}

      Thus, we have shown that $\sum\limits_{i=0}^{k+1} 2^i = 2^{(k+1)+1} - 1$.


    \vspace{1cm}
    \section{Recursive Invariants}
    
    The function \texttt{minEven}, given below in pseudocode, takes as
    input an array $A$ of size $n$ of numbers.  It returns the
    smallestest \emph{even} number in the array.  If no even numbers
    appear in the array, it returns positive infinity ($+\infty$).
    Using induction, prove that the \texttt{minEven} function works
    correctly.  Clearly state your recursive invariant at the
    beginning of your proof.

    \begin{alltt}
Function minEven(A,n)
  If n is 0 Then
    Return +infinity
  Else
    Set best To minEven(A,n-1)
    If A[n-1] < best And A[n-1] is even Then
      Set best To A[n-1]
    EndIf
    Return best
  EndIf
EndFunction
    \end{alltt}

    %%%%%%%%%%%%%%
    %%%% TODO %%%%
    %%%%%%%%%%%%%%
    % Give your recursive invariant and your proof here.
      Recursive invariant:

      \vspace{1.5mm}
      $P(n) =$ if there is even number in the array A, minEven() will return the smallest 
      even number in the array. Otherwise (if there is no even number in the array A), minEven() 
      will return positive infinity ($+\infty$). 

      \vspace{5mm}
      1. Base case: $n = 0$

      \vspace{1.5mm}
      If $n = 0$, it means the array A is empty, suggesting that there is no even number in the
      array A, and the algorithm will return $+\infty$. 

      \vspace{5mm}
      2. Inductive Hypothesis: 

      \vspace{1.5mm}
      Assume that the algorithm minEven() works correctly for all arrays
      of size n where $0 \leqslant n \leqslant k$.

      \vspace{5mm}
      3. Inductive Step: 

      Show that the algorithm minEven() works correctly for arrays with $k+1$ elements. 

      \vspace{3mm}
      minEven($A, k+1$):

      \vspace{1.5mm}
      If $k+1 > 0$, minEven($A, k+1$) will call minEven($A, k$), which by our Inductive Hypothesis
      works correctly. Thus, minEven($A, k+1$) will return the current smallest even number 
      in the array A. 

      \vspace{3mm}
      If the number at index $k$ is smaller than the current smallest even number returned by minEven($A, k$),
      which means the value at index $k$ is now the smallest even number in the array, the current smallest
      even number (stored in the variable\textit{ best }) will be updated to the value at index k.

      \vspace{1.5mm}
      Otherwise, the current smallest even value in the array A will be returned. 
\end{document}
