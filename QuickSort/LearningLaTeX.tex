% This is a LaTeX file.  LaTeX is a language which can be used to
% typeset documents similar to those one might get from
% e.g. OpenOffice or Microsoft Word.  The difference is that, while
% those are "WYSIWYG" ("what you see is what you get") tools, LaTeX is
% a markup language.  This has advantages and disadvantages, but one
% of the major advantages is in writing documents that involve a lot
% of symbols or patterns.  You are not required to use this file (or
% LaTeX in general) to complete your work, but we expect it will help.
% :)

% First and foremost: LaTeX files must be compiled like C++ programs.
% In order to compile your LaTeX file, run
%
%     pdflatex LearningLaTeX.tex
%
% This will produce a file LearningLaTeX.pdf which you may then
% view with a PDF viewer:
%
%     evince LearningLaTeX.pdf
%
% Remember that you must recompile your document if you change the
% .tex file!

% The rest of this document will contain some examples of how to use
% LaTeX.  As you may have guessed, the
% '%' character is an end-of-line comment marker, much
% like // in C++ or # in Python.  Keep reading!



% The following command is typically the first in any document.  It
% says that we want the PDF to be in an article format.  It's a
% sensible default for simple documents.  Commands in LaTeX start with
% a backslash.  Some commands take arguments, each of which is
% surrounded by a pair of braces.  So, in LaTeX,
%
%     \documentclass{article}
%
% is sort of writing this in C++:
%
%     documentclass("article");
\documentclass{article}

% Next, we need to bring in common commands, just like we would in
% Python or C++.  Rather than "import" or "#include", we use the
% "\usepackage" command.

\usepackage{amsmath} % Math notation from the American Mathematical Society.
\usepackage{amssymb} % Typographical symbols from the same.

% In addition to commands, LaTeX has an idea of "environments".  These
% are sort of like indentation in Python or the { ... } notation in
% C++, but they differ in what they do.  The most important
% environment is called "document" and everything that appears in your
% document must be inside of it.  We start an environment with the
% "begin" command and stop it with the "end" command.  Although it is
% not necessary, we typically indent the contents of the environment
% to make the .tex file easier to read.
\begin{document}

% Everything we've seen so far has been a comment or a command.  If I
% want text, all I have to do is put it inside of this document
% environment.
    Hello!
% Extra spaces in LaTeX don't matter, so that "Hello!" won't be
% indented.  Instead, it will be in paragraph.  In fact, newlines are
% treated like spaces, so the following is on the same line.
How are you?

    % If you want a new paragraph, you insert a blank line.  Since the above
    % line is blank, the following will be in its own paragraph.
    This is a new paragraph.

    % Basic text formatting is pretty simple in LaTeX.  The \textbf command
    % makes boldfaced text; the \textit command makes text italic.  For
    % instance:
    \textbf{This is bold.} \textit{This is italic.} This is plain text.
    % If you have a good editor, the above might even be rendered to
    % hint at the font changes.

    % So far, everything we have written has been in "paragraph mode".
    % It is also possible to write LaTeX in "math mode".  One way of
    % entering math mode is to write a dollar sign.  Another dollar
    % sign leaves math  mode.  This kind of math mode can be
    % written alongside text.
    
    Here's something in paragraph mode: 4+2x.
    Here it is in math mode: $4+2x$.

    % Math mode is also important because it allows you to use the
    % mathematical
    % notation from the AMS packages that we named above.  For
    % instance, the
    % command \ldots draws a sequence of dots, like an ellipse.
    $1+2+\ldots+10 = 55$

    % In math mode (but not in paragraph mode), the symbols ^ and _ have special
    % meaning: they superscript or subscript the very next thing.  For instance,
    % the following produces the letter "x" with a subscripted "i" and a
    % superscripted "10".
    $x_{i}^{10}$

    % Let's consider using those subscripts for our purposes.  We might want to
    % write a summation; conveniently, this is possible with the \sum command.
    % Fractions can be written using \frac, which takes two arguments: the
    % numerator and the denominator.  The following, for instance, is a common
    % summation:
    $\sum_{i=1}^{n} i = 1 + 2 + \ldots + n = \frac{n(n+1)}{2}$

    % That summation looks a little funny.  If we want the superscript
    % immediately above it, we can add the \limits command.  Also, we might not
    % want the numerator and denominator to shrink; we can use \dfrac instead of
    % \frac for that.
    $\sum\limits_{i=1}^{n} i = 1 + 2 + \ldots + n = \dfrac{n(n+1)}{2}$

    % This could get kind of hard to read.  If you need a little vertical
    % spacing, you can use the \vspace command.  It takes an argument describing
    % the amount of space, which can be written in different measurements.
    \vspace{5mm}

    % For Big-O proofs, we'll also want to be able to write the logical
    % quantifiers "for all" and "exists" (the upside-down A and the backwards
    % E).  Conveniently, these are accessible via the \forall and \exists
    % commands.
    $\forall x>0. \exists y<0. x+y = 0$

    % We can also get comparison symbols such as "greater than or equal to" if
    % we know the right name.
    \vspace{5mm}
    $x \geq y \leq z$

    % Finally, we'll want to be able to write out our proofs in an organized
    % fashion.  The $...$ form is good for inlining text into paragraphs,
    % but sometimes we just want a big block of math.  There's an environment
    % called "math" that does this:
    \vspace{5mm}
    \begin{math}
        % Everything inside of this environment defaults to math mode.  Within
        % this environment, you can write anything that's legal inside of the
        % $...$ delimiters.  In math mode, though, spaces and newlines are
        % handled differently; a blank line does not start a new paragraph.  To
        % make things line up nicely, we'll use an environment called "array".
        %
        % The array environment represents a table-like grid.  It takes an
        % argument describing the columns of the grid.  Each column is separated
        % with "&" and each row is separated with "\\".  For instance, the
        % following is a grid of numbers.
        \begin{array}{l l l}
            1 & 2 & 3 \\
            4 & 5 & 6 \\
            & 11 & 13 \\
        \end{array}
        % This grid is 3 columns wide because there are 3 "l"s in the argument
        % to the array environment.  "l" means "left-aligned"; we could also
        % have "c" or "r" for center or right, respectively.  The lower-left
        % corner of the grid doesn't have anything in it; we can leave cells
        % empty.
    \end{math}

    % Here's another block of math as a new paragraph!
    \vspace{5mm}
    \begin{math}
        % We can use these arrays to build proof-like layouts.  For instance,
        % I might work on a polynomial product like so:
        \begin{array}{r c l}
            (2x + 3)(x^{2} - x) & = & 2x(x^{2}) + 3(x^{2}) - 2x(x) - 3(x) \\
                                & = & 2x^{3} + 3x^{2} - 2x^{2} - 3x \\
                                & = & 2x^{3} + x^{2} - 3x \\
        \end{array}
    \end{math}

    \vspace{5mm}
    \begin{math}
        % If you need normal text inside of a math environment, one way to
        % render it is using the \textnormal command.
        x^{2} + 3x + 5 \textnormal{ is } O(x^{2})
    \end{math}

    % LaTeX is a complex language with a lot of community-written libraries and
    % other features.  Conference articles and even entire books are often
    % written in this format.  There's a lot one could learn about this system,
    % but the above should see you through CS35 and several other courses where
    % basic mathematical notation is required.
    %
    % And as an added bonus, you can keep more of your homework on your hard
    % drive instead of on paper.  :)
\end{document}
