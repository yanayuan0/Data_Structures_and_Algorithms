% This LaTeX file contains your written lab questions.  You may answer
% these questions just by inserting your answer into this document.
%
% If you're unfamiliar with LaTeX, see the document LearningLaTeX.tex
% in this same directory.  It contains a brief explanation and a few
% snippets of LaTeX code to get you started; in fact, it should have
% everything you need to complete this assignment.

\documentclass{article}

\usepackage{amsmath}
\usepackage{amssymb}
\usepackage{amsthm}
\usepackage{graphicx}

\begin{document}

    \section{QuickSort}

    See the source code file \texttt{quickSort.cpp} and the tests
    given in \texttt{tests.cpp}. No written response is needed for
    this part of the lab.

    \section{Big-O Proofs}

    \vspace{1.5mm}
    \noindent {\large\textbf{Problem 1.}} Show that $8n^3+7n^2-12$ is $O(n^3)$.

    \vspace{1.5mm}
    \noindent Assume that n is an integer and n$>0$
    \\ $8n^{3}\leqslant 8n^{3}$
    \\ $7n^{2}\leqslant 7n^{3}$
    \\ $-12<0$
    \\ $8n^3+7n^2-12 \leqslant 8n^3+7n^3 = 15n^3$
    \\ $n_{o} = 1$
    \\ Therefore f(n) is O($n^3$) when $n_{o} = 1$ and c = 15
    
    % Write your proof of the above statement here.

    \vspace{0.5cm}
    \noindent {\large\textbf{Problem 2.}} Show that $6n^2-n+4$ is $O(n^2)$.

    \vspace{1.5mm}
    \noindent Assume that n is an integer and n$>0$
    \\ $6n^{2}\leqslant 6n^{2}$
    \\ $-n<0$
    \\ $4\leqslant 4n^{2}$
    \\ $6n^2-n+4 \leqslant 6n^2+4n^2 = 10n^2$
    \\ $n_{o} = 1$
    \\ Therefore f(n) is O($n^2$) when $n_{o} = 1$ and c = 10
    % Write your proof of the above statement here.

    \vspace{0.5cm}
    \section{Mystery Functions}

    \vspace{1.5mm}
    \noindent fnA(n) big-O: O(n)
    \\ $n/2 \leqslant n$
    \vspace{1.5mm}
    \\ fnB(n) big-O: $O(n^2)$
    \\ $n^2 \leqslant n^2$
    \vspace{1.5mm}
    \\ fnC(n) big-O: $O(nlog(n))$
    \\ $nlog(n) \leqslant nlog(n)$
    \vspace{1.5mm}
    \\ fnD(n) big-O: $O(n^4)$
    \\ $n^2*n^2 \leqslant n^4$
    \vspace{1.5mm}
    \\ fnE(n) big-O: $O(1)$
    \\ $4 \leqslant 4*1$, c=4
    \vspace{1.5mm}
    \\ fnF(n) big-O: $O(n^3)$
    \\ $n^3 \leqslant n^3$
    % Give your Big-O analysis of each of the mystery functions here.
    \vspace{1.5mm}
    \\ From fastest to slowest: E, A, C, B, F, D
    % After your explanations, list the mystery functions A-E in sorted
    % order from fastest to slowest.
    
\end{document}
